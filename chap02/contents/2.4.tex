\section{2.4 复合反应}
\begin{frame}{1 复合反应}
	在同一个反应物系中同时进行若干个化学反应时,称为复合反应。
	\\由于存在多个化学反应,物系中任一反应组分既可能只参与其中一个反应,也可能同时参与其中若干个反应。
	\\某一反应既可能是某一反应的反应物,又可能是另一反应的反应产物。
	\\在这种情况下,反应进程中该组份的反应量是所参与的各个化学反应共同作用的结果。
	\\我们把单位时间内单位体积反应混合物中某一组份$i$的反应量叫做该组份的转化速率($i$为反应物)或生成速率($i$为反应产物),并以符号$R_i$表示。
\end{frame}

\begin{frame}{2 反应组份的转化速率和生成速率}
	$R_i$应等于按组份$i$计算的各个反应的反应速率的代数和。
	$$R_i=\sum_{j=1}^{M} \nu_{ij}\overline{r}_j$$
	\\$\overline{r}_j$为第$j$个反应式的反应速率,乘以组份$i$在第$j$个反应中的化学计量系数$\nu_{ij}$,则得按组份$i$计算的第$j$个反应的反应速率。
	\\$R_i$值可正可负,若为正,表示该组份在反应过程中是增加的,$R_i$代表生成速率;若为负则情况相反,$R_i$表示消耗速率,或转化速率。
	\\转化速率或生成速率$R_i$与反应速率$r_i$的区别在于前者是针对若干反应,而后者则是对一个反应而言。
	\\如果只进行一个反应,$r_i=|R_i|$
\end{frame}

\begin{frame}{3 如何由$R_i$求$\overline{r}_j$}
	复合反应动力学实验测量得到的是各个反应综合的结果,即反应组分的生成速率或消耗速率。动力学研究最关心的是各个反应的反应速率,因之就存在一个如何由$R_i$求$\overline{r}_j$的问题。
	\begin{itemize}
		\item[\bullet ] 明确系统中有哪些反应,并分清主次。
		\item[\bullet ] 忽略次要反应,只考虑起主要作用的反应。
		\item[\bullet ] 设所要考虑的反应数目为M,通过实验测定不少于M个组分的生成速率或消耗速率。
		\item[\bullet ] 得到M个线性代数方程,解之得各个反应的反应速率$\overline{r}_j$。
	\end{itemize}
	~~~~~~~~应注意,这M个反应必须是独立反应。所谓独立反应是指这些反应中任何一个反应都不可能由其余反应进行线性组合而得到。
\end{frame}


\begin{frame}{4 独立反应数的确定}
	以甲烷水蒸汽转化反应为例
	$$\mathrm{CH_4+H_2O\rightleftharpoons CO+3H_2}$$
	$$\mathrm{CH_4+2H_2O\rightleftharpoons CO_2+4H_2}$$
	$$\mathrm{CO+H_2O\rightleftharpoons CO_2+H_2}$$
	\\其中只有两个是独立反应,因为其中总有一个反应可由其余两个线性组合得到,例如$(2)-(1)=(3),(1)+(3)=(2)$。
\end{frame}


\begin{frame}{4.1 通过化学计量系数矩阵得到独立反应数}
	对于化学计量系数矩阵A,其元素由全部5个反应组分在对应的3个化学反应式中的化学计量系数构成。
	$$\mathrm{CH_4~H_2~H_2O~CO~CO_2}$$
	$$\mathrm{A} = \begin{pmatrix}
	-1&  3&  -1&  -1& 0\\
	-1&  4&  -2&  0& 1\\
	0&  1&  -1&  -1& 1
\end{pmatrix}~~~~
\begin{matrix}
	(1)\\
	(2)\\
	(3)
\end{matrix}$$
	\\经初等变换后,求得矩阵A的秩$R(\mathrm{A})=2$,此时独立反应数等于矩阵的秩。
\end{frame}


\begin{frame}{4.2 通过原子系数矩阵得到独立反应数}
	在未知化学反应式的情况下,还可根据反应组分确定独立反应数。将反应式中所包含的全部元素种类,按分别出现在5个反应组分中的原子个数,写成原子系数矩阵B。
	$$\mathrm{CH_4~H_2~H_2O~CO~CO_2}$$
	$$\mathrm{A} = \begin{pmatrix}
		-1&  3&  -1&  -1& 0\\
		-1&  4&  -2&  0& 1\\
		0&  1&  -1&  -1& 1
	\end{pmatrix}~~~~~~ 
	\begin{matrix}
		\mathrm{C}\\
		\mathrm{H}\\
		\mathrm{O}
	\end{matrix}$$
	\\求得矩B的秩$R(\mathrm{B})=3$,此时的独立反应数等于反应组分数$-R(\mathrm{B})$
	$$5-3=2$$
	\end{frame}


\begin{frame}{5 复合反应的基本类型}
	复合反应包括三个基本类型:
	\begin{itemize}
		\item[\bullet ] 并列反应\par 反应系统中各个反应的反应组分各不相同
		\item[\bullet ] 平行反应\par 反应物完全相同而反应产物不相同或不全相同
		\item[\bullet ] 连串反应\par 一个反应的反应产物同时又是另一个反应的反应物
	\end{itemize}
	~~~~~~~~可逆反应亦属复合反应之列,但它可按单一反应的办法来处理,所以不将它包括在内。
\end{frame}


\begin{frame}{5.1 并列反应}
	$$\ce{A ->P}$$
	$$\ce{B ->Q}$$
	\\各个反应都可按单一反应来处理而得到相应的速率方程。
	\\任一个反应的反应速率不受其他反应的反应组分浓度的影响,但有些催化反应除外。
	\\因此,各个反应独立进行是并列反应的动力学特点。
	\\如果是变容过程,一个反应进行的速率会受到另一个反应速率的影响,如气相反应\ce{A ->P},\ce{2B ->Q},第二个反应会改变反应物系的体积,从而改变A和P的浓度,因之影响第一个反应的速率。
\end{frame}


\begin{frame}{5.2 平行反应}
	$$~~~\ce{A ->P}\qquad r_\mathrm{P}=k_1c_\mathrm{A}^\alpha$$
	$$\ce{\nu_{A}A ->Q}\qquad r_\mathrm{Q}=k_2c_\mathrm{A}^\beta$$
	\\如果我们的目的是生产P,则P称为{\color{red}目的产物(或主产物)},其它产物均称{\color{red}副产物}。生成主产物的反应称{\color{red}主反应},其它的均称为{\color{red}副反应}。
	\\研究复合反应的目标之一是加快主反应的速率,降低副反应的速率,以获得尽可能多的目的产物。通常用{\color{red}瞬时选择性}来评价主副反应速率相对大小。
	$$S=\mu_\mathrm{PA}\frac{\mathscr{R}_\mathrm{P}}{|\mathscr{R}_\mathrm{A}|}$$
	\\式中,$\mu_\mathrm{PA}$为生成1molP所消耗的A的mol数。用瞬时选择性这个词,是要表明其值随反应物系的组成及温度而变,它是一个瞬时值。
\end{frame}


\begin{frame}{5.2.1 浓度对平行反应瞬时选择性的影响}
	$$S=\frac{k_1c_\mathrm{A}^\alpha}{k_1c_\mathrm{A}^\alpha+|{\nu_\mathrm{A}}|k_2c_\mathrm{A}^\beta}=\frac{1}{1+\frac{k_2}{k_1}|{\nu_\mathrm{A}}|c_\mathrm{A}^{\beta -\alpha}}$$
	\\若温度一定,
\end{frame}